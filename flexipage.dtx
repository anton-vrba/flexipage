% \iffalse meta-comment
%
% Copyright (C) 2022 by <+author+> <<+email+>>
% ---------------------------------------------------------------------------
% This work may be distributed and/or modified under the
% conditions of the LaTeX Project Public License, either version 1.3
% of this license or (at your option) any later version.
% The latest version of this license is in
%   http://www.latex-project.org/lppl.txt
% and version 1.3 or later is part of all distributions of LaTeX
% version 2005/12/01 or later.
%
% This work has the LPPL maintenance status `maintained'.
%
% The Current Maintainer of this work is <+maintainer+>.
%
% This work consists of the files flexipage.dtx and flexipage.ins
% and the derived filebase flexipage.sty.
%
% \fi
%
% \iffalse
%<*driver>
\ProvidesFile{flexipage.dtx}
%</driver>
%<package>\NeedsTeXFormat{LaTeX2e}[2020/10/01]
%<package>\ProvidesPackage{flexipage}
%<*package>
    [2022/05/28 1.0 Easy setup for page geometry]
%</package>
%
%<*driver>
\documentclass{ltxdoc}
\usepackage{xspace}
\usepackage{tikz,amsmath, amssymb,bm,color}
\usetikzlibrary{shapes,arrows}
\usetikzlibrary{calc}
\EnableCrossrefs
\CodelineIndex
\RecordChanges
\begin{document}
  \voffset = -45pt
  \textheight=210mm
  \DocInput{flexipage.dtx}
  \PrintChanges
%  \PrintIndex
\end{document}
%</driver>
% \fi
%
%
%
% \CharacterTable
%  {Upper-case    \A\B\C\D\E\F\G\H\I\J\K\L\M\N\O\P\Q\R\S\T\U\V\W\X\Y\Z
%   Lower-case    \a\b\c\d\e\f\g\h\i\j\k\l\m\n\o\p\q\r\s\t\u\v\w\x\y\z
%   Digits        \0\1\2\3\4\5\6\7\8\9
%   Exclamation   \!     Double quote  \"     Hash (number) \#
%   Dollar        \$     Percent       \%     Ampersand     \&
%   Acute accent  \'     Left paren    \(     Right paren   \)
%   Asterisk      \*     Plus          \+     Comma         \,
%   Minus         \-     Point         \.     Solidus       \/
%   Colon         \:     Semicolon     \;     Less than     \<
%   Equals        \=     Greater than  \>     Question mark \?
%   Commercial at \@     Left bracket  \[     Backslash     \\
%   Right bracket \]     Circumflex    \^     Underscore    \_
%   Grave accent  \`     Left brace    \{     Vertical bar  \|
%   Right brace   \}     Tilde         \~}
%
%
% \changes{<+version+>}{<+date+>}{Converted to DTX file}
%
% \DoNotIndex{\newcommand,\newenvironment}
%
%
% \providecommand*{\url}{\texttt}
% \providecommand\lgtarg[1]{{\ttfamily<}\meta{#1}{\ttfamily>}}
% \GetFileInfo{flexipage.dtx}
% \title{The \textsf{flexipage} package}
% \author{Anton Vrba}
% \date{\fileversion~from \filedate}
%
%
% \maketitle
%
%  \begin{abstract}
% \noindent The package flexipage allows easy page layout if marginalia is required.
% Mid document changes are possible such as: new marginal width, full width text,
% and landscape text without marginal. Partners well with the package \textsf{sidenotesplus}.
% The package also aids the layout for book printing, allowing for binding corrections
% and adding page bleed, if required,
% \end{abstract}
%
% \section{Introduction}
%
% The aim is an easy to use the page geometry setup,
% and allow mid-document changes such as from portrait
% to landscape, or change the text body width to marginalia width ratio.
% The philosophy is simple, easy to understand options quickly set up the page geometry:
% \begin{enumerate}
%   \item |pageheight| and |pagewidth| of final document, default inherited.
%   \item Defines page margins |inner|, |outer|, |top| and |bot|. Inner refers to the binding
%         that is the left side margin of on odd pages in |twoside| documents, or
%         of pages in |oneside| documents. Page margins surround the printable area.
%         Default for all |36pt|, or |0.5in|.
%   \item Add a |binding| correction, default |9pt|.
%   \item Add an |outerbleed|, increases the papersize by adding a bleed area to the
%         non bound margins. For the seldom requirement an |innerbleed| adds to the
%         paperwidth on the inner side.
%         Refer to your publisher. Default for both |0pt|.
%   \item Divide the remaining printable area into a vertical |grid|, default |12|.
%   \item Insert a |marginparsep| (default |3ex|) at the |margingrid| position
%         counting from the outer edge; 0 is for no
%         marginalia, 1 is a narrow marginal note area, and 4 is a marginal note area
%         of one halve the width of the text body area's width when |grid=12|.
%         The marginal area is placed by the outer edge.
%   \item The \LaTeX\xspace |headsep|, |headheight| and |footskip| dimension
%         can also be passed as options, otherwise default to class values.
%   \item As a visual guide, the option |show| frames the page geometry in the output
% \end{enumerate}
%
% \footnotesize
% \noindent \begin{tikzpicture}
%
% \draw [fill=yellow!30]  (-4,4.4) node (v1) {} rectangle (3,-4.2);
% \draw[fill=white]  (-3.8,4.2) rectangle (2.8,-4);
% \draw [color=green, fill=green!20 ] (-2.8,3.6) rectangle (2.2,-3.4);%
% \draw [fill=orange!10]  (-2.8,3.6) rectangle (0.8,3.4);
% \draw [fill=orange!10] (1,3) rectangle (2.2,-2.8);
% \draw  [fill=orange!10](-2.8,3) rectangle (0.8,-2.8);
% \draw  [fill=violet!10] (-3.8,-4.2) rectangle (v1);

% \draw [loosely dashed, blue] (-3.4,4.2) -- (-3.4,-4);
% \draw (-2.8,-3.4) -- (0.8,-3.4);
% \draw [loosely dashed] (1.8,3) -- (1.8,-2.8);
% \draw [loosely dashed] (1.4,3) -- (1.4,-2.8);
% \draw [loosely dashed] (0.4,3) -- (0.4,-2.8);
% \draw [loosely dashed] (0,3) -- (0,-2.8);
% \draw [loosely dashed] (-0.4,3) -- (-0.4,-2.8);
% \draw [loosely dashed] (-0.8,3) -- (-0.8,-2.8);
% \draw [loosely dashed] (-1.2,3) -- (-1.2,-2.8);
% \draw  [loosely dashed] (-1.6,3) -- (-1.6,-2.8);
% \draw [loosely dashed](-2,3) -- (-2,-2.8);
% \draw [loosely dashed](-2.4,3) -- (-2.4,-2.8);
% \node at (-1.2,3.5) {Header};
% \node at (-0.9,0.3) {Text Body};
% \node [rotate=90] at (1.6,0.3) {Marginal};
% \node [text width=65mm, right ] at (3.5,0.1)
% {\small  The basic paper geometry for book printing is illustrated to the left.
% This for odd page numbers, the recto page. The even pages mirror this geometry
% to maintain verso--recto symmetry.
% The final page size is the white area. The yellow outer coloured area is the
% |outerbleed| and the violet colored area on the left is the |innerbleed|, these
% are trimmed and discarded. Please refer to your publisher if needed.
%
% ~~~The green area is the text block containing the text body, the marginal,
% the header and footer and is defined by the margins |top|,
% |outer| and |bottom| to the trimmed paper's edge.
% The inner side (left) vertical dot-dashed line represents the |binding| correction and the |inner|
% margin is defined from this line.
%
% ~~~Here the illustrated text width plus the marginal width are divided into 12 equal |grid=12| divisions
% with the marginal takin 3 of them, defined by |marginalgrid=3|.
% .};
% \node at (-1.2,-3.3) {Footer};
% \end{tikzpicture}\small
% \\ Also refer to:
% \url{https://www.overleaf.com/learn/latex/Single\_sided\_and\_double\_sided\_documents}
% \normalsize
%
% \section{Usage}
% \subsection{Document preamble}
% \small
% \begin{verbatim}
% \PassOptionsToPackage{
% % a4paper         , % or another recognised paper size
% % paperheight =   , %  inherits from class
% % paperwidth =    , %  inherits from class
% % inner =         , %  default 36pt or 0.5in
% % outer =         , %          36pt or 0.5in
% % top =           , %          36pt or 0.5in
% % bot =           , %          36pt or 0.5in
% % binding =       , %           9pt or 0.125in
% % outerbleed =    , %           0pt, but check with your printer
% % innerbleed =    , %           0pt, but check with your printer
% % grid =          , %          12 it is just a number
% % marginalgrid =  , %           3 it is just a number
% % marginparsep =  , %           3ex, if using sidenotesplus change to 4 or 5ex
% % headsep =       , % inherits from class
% % headheight =    , % inherits from class
% % footskip=       , % inherits from class
% % show            , % a switch to show frames around the printable areas.
% % fancyhdrOff     , % Do not use the fancyhdr package
% % NoFixMargin       % a possible future requirement
% }{flexipage}
% \usepackage{flexipage}
%   \end{verbatim}
% \normalsize\vspace{-20pt}
%
% \paragraph{Known Issue:} If using the option |show| and the package \textsf{xcolor}
% is loaded later with options, an error will occur. The reason
% is that when option |show| is specified the package \textsf{eso-pic} is loaded
% whch in turns loads \textsf{xcolor} without options.
%
% \paragraph{Marginalia Position Issue:} When doing mid document changes to |\textwidth|
% and |\marginparwidth| then |\marginpar{content}| renders the marginalia to the correct
% width but places it at the old position. A patch is provided here
% to correct this issue. Should this ever be fixed in the
% latex base software then the |NoFixMargin| option would be required.
%
% \paragraph{Using \textsf{fancyhdr} package} By default the \textsf{fancyhdr} package
% is loaded and the |\headerlength| is set to span the page, that is it overhangs the
% text body. The advantage here is that the header has the same format throughout
% the document if mid-document layout changes are made. If no such changes are
% made then use the |fancyhdrOff| and load \textsf{fancyhdr} in your preamble. Now
% the header does not overhang the text body and is as shown in the previous illustration.
%
%
% The following paper sizes are recognised:
%
% ~
%
% \begin{tabular}{rlp{2ex}rl}
% \multicolumn{2}{c}{Standard Paper} & &\multicolumn{2}{c}{Standard Book}  \\
%        a4paper &210$\times$297mm & &   US-Trade &6$\times$9in \\
%        a5paper &148$\times$210mm & & UK-B-Format &135$\times$210mm \\
%        b4paper &250$\times$353mm & & US-B-Format &129$\times$198mm \\
%        b5paper &176$\times$250mm & &   UK-Royal &156$\times$234mm \\
%        c4paper &229$\times$324mm & &     Quarto &9.5$\times$12in \\
%        c5paper &162$\times$229mm & &     Octavo &6$\times$9in \\
%       b4jpaper &257$\times$364mm & & CrownQuarto &189$\times$246mm \\
%       b5jpaper &182$\times$257mm & &       Demi &138$\times$216mm \\
%    letterpaper &8.5$\times$11in \\
%     legalpaper &8.5$\times$14in \\
% executivepaper &7.25$\times$10.5in \\
% \end{tabular}

%
%\subsection{User Interfaces}
%
%
% \DescribeMacro{\NewMarginPage} Changes text--marginal ratio.
%  |\NewMarginPage|\oarg{g}\lgtarg{parsep}\marg{m}
% where optionally the grid count is reset to |grid|=\textit{g}, and marginal  width is
% determined by |marginalgrid|=\textit{m}. The ratio of the lengths |\textwidth|
% to |\marginparwidth| is $g-m$ to $n$. The second option changes the |\parsep| length.
%
%
% \DescribeMacro{\NewFullPage} Continues the document with a new page without margins,
% and is equivalent to |\NewMarginPage|\marg{0}
%
%
%
% \DescribeMacro{\OldMarginPage}
% \DescribeMacro{\ResetFlexiPage} After |\NewMarginPage| or |\NewFullPage| revert to
% the previous page geometry, these are not stacked. |\ResetFlexiPage| reverts to the layout specified in
% the initial options
%
%
%
%
% \DescribeEnv{Landscape}
% \DescribeEnv{landscape}
% These two environments provide landscape layouts. The big |Landscape| provides
% full text block landscape. The small |landscape| environments respects the
% portrait orientated marginal space which now is white space. Do not use
% marginalia in landscape mode.
%
% \DescribeMacro{\Landscape}
% \DescribeMacro{\endLandscape}
% \DescribeMacro{\landscape}
% \DescribeMacro{\endlandscape} Alternatives to |\begin{Landscape}| $\cdots$  $\cdots$ |\end{Landscape}|
% %
%
% \section{Implementation}
%
% \iffalse
%<*package>
% \fi
%
%    \begin{macrocode}
\RequirePackage{xparse}   % new LaTeX3 syntax to define macros and environments
\RequirePackage{l3keys2e} % nice way for option processing
\RequirePackage{calc}     % length manipulation
\RequirePackage{fp}       % floating point calculation
\RequirePackage{ifoddpage}
\RequirePackage{graphics}  % provides landscaping
\RequirePackage{mparhack} % not needed here, but the user certainly needs it.
\RequirePackage{etoolbox} % provides \patchcmd
% ****** \RequirePackage{eso-pic} only loaded later if option show is specified
\newlength{\fleximarginparsep}
\newlength{\flx@paperheight}
\newlength{\flx@paperwidth}
\newlength{\flx@temp}
\newlength{\flx@LSfix}
\setlength{\flx@LSfix}{0pt}
\newlength{\flx@lsfix}
\setlength{\flx@lsfix}{0pt}
\newlength{\flx@textheigt}
\newlength{\flx@orgtextwidth}
\newlength{\flx@oldparsep}
\newlength{\flx@oparsep}
\newlength{\flx@binding}
\newlength{\marginparsepodd}
\newlength{\marginparsepeven}
\newlength{\flx@inner}
\newlength{\flx@outer}
\newlength{\flx@top}
\newlength{\flx@bot}
\newlength{\flx@parsep}
\newlength{\flx@printarea}
\newlength{\flx@outerbleed}
\newlength{\flx@innerbleed}
\setlength{\flx@binding}  {9pt}
\setlength{\flx@inner} {36pt}
\setlength{\flx@outer} {36pt}
\setlength{\flx@top}   {36pt}
\setlength{\flx@bot}   {45pt}
\setlength{\flx@parsep}{3ex}
\setlength{\flx@outerbleed} {0pt}
\setlength{\flx@innerbleed} {0pt}
\FPset\flx@m{3}
\FPset\flx@g{12}
\newif\ifSF@frame\SF@framefalse
\newif\ifFix@Margin\Fix@Margintrue
\newif\ifUse@fancyhdr\Use@fancyhdrtrue
%    \end{macrocode}
%
%
%
%    \begin{macrocode}
\setlength{\flx@paperheight}{\paperheight}
\setlength{\flx@paperwidth}{\paperwidth}

\NewDocumentCommand \flx@setpapersize {o o m}{
\setlength{\flx@paperheight}{#2#3}
\setlength{\flx@paperwidth}{#2#3}
}


\ExplSyntaxOn
\keys_define:nn { flexiprint }
{
        binding .code:n = \setlength{\flx@binding}{#1},
          inner .code:n = \setlength{\flx@inner}{#1},
          outer .code:n = \setlength{\flx@outer}{#1},
            top .code:n = \setlength{\flx@top}{#1},
            bot .code:n = \setlength{\flx@bot}{#1},
   marginparsep .code:n = \setlength{\flx@parsep}{#1},
    paperheight .code:n = \setlength{\flx@paperheight}{#1},
     paperwidth .code:n = \setlength{\flx@paperwidth}{#1},
     outerbleed .code:n = \setlength{\flx@outerbleed}{#1},
     innerbleed .code:n = \setlength{\flx@innerbleed}{#1},
           grid .code:n = \FPset\flx@g{#1},
   marginalgrid .code:n = \FPset\flx@m{#1} ,
           show .code:n = \SF@frametrue ,
    NoFixMargin .code:n = \Fix@Marginfalse ,
        headsep .code:n = \setlength{\headsep}{#1},
     headheight .code:n = \setlength{\headheight}{#1},
       footskip .code:n = \setlength{\footskip}{#1},
    fancyhdrOff .code:n = \Use@fancyhdrfalse,
        a4paper .code:n = \flx@setpapersize[210][297]{mm},
        a5paper .code:n = \flx@setpapersize[148][210]{mm},
        b4paper .code:n = \flx@setpapersize[250][353]{mm},
        b5paper .code:n = \flx@setpapersize[176][250]{mm},
        c4paper .code:n = \flx@setpapersize[229][324]{mm},
        c5paper .code:n = \flx@setpapersize[162][229]{mm},
       b4jpaper .code:n = \flx@setpapersize[257][364]{mm},
       b5jpaper .code:n = \flx@setpapersize[182][257]{mm},
    letterpaper .code:n = \flx@setpapersize[8.5][11]{in},
     legalpaper .code:n = \flx@setpapersize[8.5][14]{in},
 executivepaper .code:n = \flx@setpapersize[7.25][10.5]{in},
      10x8paper .code:n = \flx@setpapersize[8][10]{in},
       US-Trade .code:n = \flx@setpapersize[6][9]{in},
    UK-B-Format .code:n = \flx@setpapersize[135][210]{mm},
    US-B-Format .code:n = \flx@setpapersize[129][198]{mm},
       UK-Royal .code:n = \flx@setpapersize[156][234]{mm},
         Quarto .code:n = \flx@setpapersize[9.5][12]{in},
         Octavo .code:n = \flx@setpapersize[6][9]{in},
    CrownQuarto .code:n = \flx@setpapersize[189][246]{mm},
           Demi .code:n = \flx@setpapersize[138][216]{mm},
}
\ProcessKeysOptions { flexiprint }

\DeclareExpandableDocumentCommand{\IfNoValueOrEmptyTF}{ m m m }
{
 \IfNoValueTF{#1}
  {#2}
  {\tl_if_empty:nTF {#1} {#2} {#3}}
}
\ExplSyntaxOff

\FPset\flx@oldg{\flx@g}
\FPset\flx@oldm{\flx@m}
\FPset\flx@og{\flx@g}
\FPset\flx@om{\flx@m}
\setlength{\flx@oparsep}{\flx@parsep}
\ifUse@fancyhdr
   \RequirePackage{fancyhdr}
   \pagestyle{fancy}
\fi

\setlength{\flx@oldparsep}{\flx@parsep}
\setlength{\flx@oparsep}{\flx@parsep}
\addtolength{\flx@binding}{\flx@innerbleed}
\global\hoffset=0pt
\global\voffset=0pt

%    \end{macrocode}
%
% \begin{macro}{\flx@setpagevertical}
%    \begin{macrocode}
\NewDocumentCommand \flx@setpagevertical {} {
  \setlength{\flx@temp}{\flx@paperwidth+\flx@innerbleed+\flx@outerbleed}
  \global\paperwidth=\flx@temp

  \setlength{\flx@temp}{\flx@paperheight+2\flx@outerbleed}
  \global\paperheight=\flx@temp

  \setlength{\flx@temp}{\flx@top+\flx@outerbleed-72pt}
  \setlength{\topmargin}{\flx@temp}

  \setlength{\flx@temp}{\paperheight - \flx@top %
            - \headheight - \headsep -\footskip - \flx@bot - 2\flx@outerbleed}
  \setlength{\textheight}{\flx@temp}
}
%    \end{macrocode}
% \end{macro}
%
%    \begin{macrocode}
\flx@setpagevertical
\setlength{\flx@textheigt}{\textheight}
\newlength{\flx@linewidth}
%    \end{macrocode}
%
% \begin{macro}{\flx@setpage}
%    \begin{macrocode}

\NewDocumentCommand \flx@setpage {} {
    \FPifgt\flx@m0\relax\else\setlength{\flx@parsep}{0pt}\fi
    \setlength{\flx@temp}{\flx@binding+\flx@inner-72pt}
    \global\oddsidemargin=\flx@temp

    \setlength{\flx@linewidth}{\paperwidth-\oddsidemargin-\flx@outer-72pt-\flx@outerbleed}
    \ifUse@fancyhdr
       \setlength{\headwidth}{\flx@linewidth}
    \fi
    \setlength{\flx@printarea}{\flx@linewidth-\flx@parsep}
    \FPsub\flx@t\flx@g\flx@m
    \setlength{\flx@temp}{\flx@printarea * \real{\flx@t}/\real{\flx@g} }
    \global\textwidth=\flx@temp

    \setlength{\flx@temp}{\flx@printarea * \real{\flx@m}/\real{\flx@g} }
    \global\marginparwidth=\flx@temp
    \global\marginparsep=\flx@parsep
    \if@twoside
      \setlength{\flx@temp}{\flx@outer+\marginparwidth+\marginparsep-1in+\flx@outerbleed}
      \global\evensidemargin=\flx@temp
    \else
      \global\evensidemargin=\oddsidemargin
    \fi
    \global\fleximarginparsep=\marginparsep
    \global\hsize=\textwidth
}
%    \end{macrocode}
% \end{macro}
%
%    \begin{macrocode}
\flx@setpage
\flx@orgtextwidth=\textwidth
\FPset\flx@oldg\flx@g
\FPset\flx@oldm\flx@m
\setlength{\flx@oparsep}{\flx@parsep}
\setlength{\flx@oldparsep}{\flx@parsep}
%    \end{macrocode}
%
%
% \begin{macro}{\NewMarginPage}
%    \begin{macrocode}
\NewDocumentCommand \NewMarginPage {o d<> m} {
   \FPset\flx@oldg\flx@g
   \FPset\flx@oldm\flx@m
   \setlength{\flx@oldparsep}{\flx@parsep}
   \ResetFlexiPage
   \IfNoValueOrEmptyTF{#2}{\relax}{\setlength{\flx@parsep}{#2} }
   \IfNoValueOrEmptyTF{#1}{\relax}{\FPset\flx@g{#1}}
   \FPset\flx@m{#3}
   \flx@setpage
   \flx@marginparcorrection
}
%    \end{macrocode}
% \end{macro}
%
%
% \begin{macro}{\OldMarginPage}
%    \begin{macrocode}
\NewDocumentCommand \OldMarginPage {} {
   \FPset\flx@m\flx@oldm
   \FPset\flx@g\flx@oldg
   \flx@ResetFlexiPage
   \setlength{\flx@parsep}{\flx@oldparsep}
   \flx@setpage
   \flx@marginparcorrection
}
%    \end{macrocode}
% \end{macro}
%
%
% \begin{macro}{\NewFullPage}
%    \begin{macrocode}
\NewDocumentCommand \NewFullPage {}{
   \FPset\flx@oldg\flx@g
   \FPset\flx@oldm\flx@m
   \setlength{\flx@oldparsep}{\flx@parsep}
   \flx@ResetFlexiPage
   \FPset\flx@m{0}
   \setlength{\flx@parsep}{0pt}
   \flx@setpage
}
%    \end{macrocode}
% \end{macro}
%
%
% \begin{macro}{\ResetFlexiPage}
%    \begin{macrocode}
\NewDocumentCommand \ResetFlexiPage{} {
   \FPset\flx@oldg\flx@g
   \FPset\flx@oldm\flx@m
   \flx@ResetFlexiPage
}
%    \end{macrocode}
% \end{macro}
%
%
% \begin{macro}{\flx@ResetFlexiPage}
%    \begin{macrocode}
\NewDocumentCommand \flx@ResetFlexiPage{} {
   \clearpage
   \FPset\flx@g\flx@og
   \FPset\flx@m\flx@om
   \setlength{\flx@parsep}{\flx@oparsep}
   \flx@setpage
   \flx@marginparcorrection
}
%    \end{macrocode}
% \end{macro}%
%
% \begin{macro}{\Landscape}
%    \begin{macrocode}
\NewDocumentCommand \Landscape {}
%    \end{macrocode}
% \end{macro}
%
%    \begin{macrocode}
{
   \FPset\flx@oldm\flx@m
   \FPset\flx@oldg\flx@g
   \setlength{\flx@oldparsep}{\flx@parsep}
  \clearpage
  \setlength{\flx@LSfix}{\textheight-\flx@linewidth}
  \FPset\flx@m{0}
  \setlength{\flx@parsep}{0ex}
  \flx@setpage
  \flx@landscape
}
%    \end{macrocode}
%
% \begin{macro}{\endLandscape}
%    \begin{macrocode}
\NewDocumentCommand \endLandscape {}
{
   \flx@endlandscape
   \FPset\flx@m\flx@oldm
   \FPset\flx@g\flx@oldg
   \setlength{\flx@parsep}{\flx@oldparsep}
   \flx@setpage
   \setlength{\flx@LSfix}{0pt}
   \flx@marginparcorrection
}
%    \end{macrocode}
% \end{macro}
%
%
% \begin{macro}{\landscape}
%    \begin{macrocode}
\NewDocumentCommand \landscape {}
%    \end{macrocode}
% \end{macro}
%
%    \begin{macrocode}
{
   \FPset\flx@oldm\flx@m
   \FPset\flx@oldg\flx@g
   \setlength{\flx@oldparsep}{\flx@parsep}
  \clearpage
  \setlength{\flx@LSfix}{\textheight-\textwidth}
  \FPset\flx@m{0}
  \setlength{\flx@parsep}{0ex}
  \flx@landscape
}
%    \end{macrocode}
%
% \begin{macro}{\endlandscape}
%    \begin{macrocode}
\NewDocumentCommand \endlandscape {}
{
   \flx@endlandscape
   \FPset\flx@m\flx@oldm
   \FPset\flx@g\flx@oldg
   \setlength{\flx@parsep}{\flx@oldparsep}
   \flx@setpage
   \setlength{\flx@LSfix}{0pt}
   \flx@marginparcorrection
}
%    \end{macrocode}
% \end{macro}
%
%
%  If the page layout includes a marginal these lscape.sty package
%  respect these
%
%
%    \begin{macrocode}
%

\def\flx@LS@rot{%
  \setbox\@outputbox\vbox{\hbox{\rotatebox{90}{\box\@outputbox}}}}

\def\flx@landscape{%
  \let\flx@LS@makecol=\@makecol
  \let\flx@LS@makefcolumn=\@makefcolumn
  \clearpage
  \begingroup
  \global\vsize=\textwidth
  \global\hsize=\textheight
  \linewidth=\hsize
  \columnwidth=\hsize
  \@colroom=\vsize
  \textheight=\vsize
  \@colht=\vsize
  \def\@makecol{\flx@LS@makecol\flx@LS@rot}%
  \def\@makefcolumn##1{\flx@LS@makefcolumn{##1}\flx@LS@rot}}

\def\flx@endlandscape{%
  \clearpage
  \endgroup
  \global\@colht=\textheight
  \global\vsize=\textheight
  \global\@colroom=\textheight}
%
%    \end{macrocode}
%
%
%
% \begin{macro}{\flx@marginparcorrection }
%
% Mid-document changes to textwidth are not communicated the LaTeX marginpar
% routines hence the patch to correct the odd-side marginparsep.
%
% lifted from https://tex.stackexchange.com/questions/174605
%
% % \end{macro}
%
%
%    \begin{macrocode}
\NewDocumentCommand\flx@marginparcorrection{}{
\ifFix@Margin
   \setlength{\marginparsepodd}{\marginparsep-\flx@orgtextwidth+\textwidth}
   \setlength{\marginparsepeven}{\marginparsep}
   \patchcmd{\@addmarginpar}%In this command
   {\mph@orig@addmarginpar}%...replace this...
   {\if@twoside\ifodd\c@page\relax%...with this
         \marginparsep=\marginparsepodd%Page isodd
      \else%twoside and eveb
         \marginparsep=\marginparsepeven%Page iseven
      \fi
      \else% oneside
         \marginparsep=\marginparsepodd
      \fi
   \mph@orig@addmarginpar}
   {}%success
   {\message{Error! Couldn't hook into command   %failure
   `\string\@addmarginpar'}} %end patch command

   {
   \def\patcherr{%
      \message{Error!Couldn't hook into command`\string\@mn@@@marginnote'}}
      \def\patchok{%
      \let\patcherr\relax%Only display error if first patch fails
      \patch%Now patch again.
   }
   \def\patch{
      \patchcmd{\@mn@@@marginnote}%In this command
         {\kern\marginparsep}%...replace this...
      {\ifx\@mn@currpage\relax\else%...with this
         \if@twoside\ifodd\@mn@currpage\relax
            \kern\marginparsepodd%Page is odd
         \else
            \kern\marginparsepeven
         \fi
         \else% one side
            \kern\marginparsepodd
         \fi
      \fi}
      {\message{Patched!}\patchok}%success(recurse)
      {\patcherr}%fail
   }
   \message{Patching`\string\@mn@@@marginnote`!}
   \patch
   \global\let\@mn@@@marginnote\@mn@@@marginnote%Makepatch global
   }
\fi
}
%    \end{macrocode}
%
%
% The package showframe.sty is adapted and localised
%
%
%    \begin{macrocode}
\ifSF@frame\RequirePackage{eso-pic}\fi  % so that below code can draw frames

\ifSF@frame\AddToShipoutPictureBG{\flx@ShowFramePicture}\fi

\newcommand*\flx@framecolor{\color{green}}
\newcommand*\flx@frameline{.8pt}
\newdimen\SF@tempdima

\newcommand\flx@ShowFramePicture{%
  \begingroup
    \color{orange}\linethickness{\flx@frameline}%
    \if@twoside
       \ifodd\c@page
          \AtPageUpperLeft{%
            \put(\flx@innerbleed,-\paperheight+\flx@outerbleed){%
              \framebox(\paperwidth-\flx@outerbleed-\flx@innerbleed,\paperheight-2\flx@outerbleed){}}}%
        \else
          \AtPageUpperLeft{%
            \put(\flx@outerbleed,-\paperheight+\flx@outerbleed){%
              \framebox(\paperwidth-\flx@outerbleed-\flx@innerbleed,\paperheight-2\flx@outerbleed){}}}%
        \fi
    \else
       \AtPageUpperLeft{%
         \put(\flx@innerbleed,-\paperheight+\flx@outerbleed){%
           \framebox(\paperwidth-\flx@outerbleed-\flx@innerbleed,\paperheight-2\flx@outerbleed){}}}%
    \fi
    \flx@framecolor\linethickness{\flx@frameline}%
    \AtTextLowerLeft{%
      \put(0,-\flx@LSfix){\framebox(\textwidth,\textheight+\flx@LSfix){}}}%

\ifUse@fancyhdr
    \if@twoside
        \ifodd\c@page
            \AtPageUpperLeft{%
               \put(\flx@inner+\flx@binding,-\flx@outerbleed-\flx@top-\headheight){%
                    \framebox(\flx@linewidth,\headheight){}}}%
            \AtPageUpperLeft{%
               \put(\flx@inner+\flx@binding,%
                       -\flx@outerbleed-\flx@top-\headheight-\headsep-\textheight%
                       -\footskip -\flx@LSfix){%
                    \line(1,0){\flx@linewidth}}}%
        \else
        \AtPageUpperLeft{%
           \put(\flx@outer+\flx@outerbleed,-\flx@outerbleed-\flx@top-\headheight){%
            \framebox(\flx@linewidth,\headheight){}}}%
        \AtPageUpperLeft{%
           \put(\flx@outer+\flx@outerbleed,%
                   -\flx@outerbleed-\flx@top-\headheight-\headsep-\textheight%
                   -\footskip -\flx@LSfix){%
                \line(1,0){\flx@linewidth}}}%

        \fi
    \else
        \AtPageUpperLeft{%
           \put(\flx@inner+\flx@binding,-\flx@outerbleed-\flx@top-\headheight){%
            \framebox(\flx@linewidth,\headheight){}}}%
        \AtPageUpperLeft{%
           \put(\flx@inner+\flx@binding,%
                   -\flx@outerbleed-\flx@top-\headheight-\headsep-\textheight%
                   -\footskip -\flx@LSfix){%
                \line(1,0){\flx@linewidth}}}%
    \fi
%
\else
    \AtTextUpperLeft{%
      \put(0,\headsep){%
        \framebox(\textwidth,\headheight){}}}%
    \AtTextLowerLeft{%
      \put(0,-\footskip-\flx@LSfix){%
        \line(1,0){\textwidth}}}%
\fi
%
\FPifgt\flx@m 0{%
    \AtTextLowerLeft{%
      \SF@tempdima=\textwidth\advance\SF@tempdima\marginparsep%
      \if@twoside
        \ifodd\c@page\else
          \if@mparswitch
            \SF@tempdima=-\marginparsep\advance\SF@tempdima-\marginparwidth
          \fi
        \fi
      \fi
      \put(\SF@tempdima,0)%
        {\framebox(\marginparwidth,\textheight){}}%
    }%
  }\fi
  \endgroup
}
%    \end{macrocode}
%
%
% \iffalse
%</package>
% \fi
%
% \Finale
\endinput
